\documentclass{article}

% Language setting
% Replace `english' with e.g. `spanish' to change the document language
\usepackage[english,ukrainian]{babel}

% Set page size and margins
% Replace `letterpaper' with `a4paper' for UK/EU standard size
\usepackage[letterpaper,top=2cm,bottom=2cm,left=3cm,right=3cm,marginparwidth=1.75cm]{geometry}

% Useful packages
\usepackage{amsmath}
\usepackage{graphicx}
\usepackage[colorlinks=true, allcolors=blue]{hyperref}
\paperwidth=210mm
\paperheight=297mm

\hoffset=0.0mm 		% By default offset from paper edge is 1 inch, this will set it to 25mm
\voffset=-5.4mm		% By default offset from paper edge is 1 inch, this will set it to 20mm

\oddsidemargin=0mm	% This will add nothing to hoffset
\evensidemargin=0mm	% This will add nothing to hoffset
\topmargin=0mm
\headheight=0mm
\headsep=0mm

\textwidth=170mm
\textheight=237mm

% \title{137. Вплив радіації на людський організм. Дозиметрія.}
% \author{\textsl {М.C.~Кухарчук$^{1}$}}

\begin{document}

\title{137. Вплив радіації на людський організм. Дозиметрія.}
\author{\textsl {М.C.~Кухарчук$^{1}$}}
\date{\vspace*{-6ex}}
\maketitle

\begin{center}
    {\small $^{1}$Група ІПС-31, курс 3, факультет комп'юторних наук та кібернетики\\
    {\tt maria.kuharchuk83@gmail.com}}
\end{center}

\section{Радіація}
\subsection{У фізиці}

У фізиці радіація – це те саме, що й випромінювання: від найдовших радіохвиль, які передають сигнали на довгі відстані, до найкоротших гамма-променів, які, якщо вірити коміксам, створили Галка. У побуті ми ж звикли називати радіацією випромінювання від розпаду ядер атомів, яке є потенційно небезпечним – його називають іонізуючим, ядерним чи радіоактивним. Для зручності далі під словом «радіація» будемо розуміти саме це значення.

У природі зустрічаються нестабільні елементи – радіонукліди – які випромінюють радіацію. Потоки часточок із космосу (космічне випромінювання), частина сонячного випромінювання, радіонукліди в довкіллі так само є радіоактивними і складають природний радіаційний фон. Радіоактивні часточки також можна синтезувати штучно – в процесі наукових досліджень, роботі ядерної галузі тощо.

Під впливом радіації матеріали можуть самі ставати радіоактивними, хімічні зв’язки у них – послаблюватися, змінюючи їх властивості, хімічні елементи – перетворюватися на інші. Радіоактивне опромінення клітин живих організмів змінює їхню здатність відновлюватися, що може призвести до загибелі, пошкодження або неправильного відновлення. Воно також може спричинити мутації у ДНК, які, якщо не відновляться, врешті призводять до розвитку пухлин.

Високі дози радіації, отримані за короткий проміжок часу від контакту з радіоактивними матеріалами, призводять до серйозних наслідків – опіків, гострої променевої хвороби (ГПХ), численних патологій, що можуть проявитися протягом тривалого часу, і навіть смерті. Після аварії на Чорнобильській АЕС лише від наслідків ГПХ загинули 44 людини. Сотні тисяч ліквідаторів, що працювали там у наступні роки, відчули погіршення здоров’я майже за усіма класами хвороб, зростання захворюваності на рак щитоподібної залози, лейкемію, пухлини, психічні та ендокринні розлади – і ще безліч проблем, які зачепили не лише їх, а їхніх нащадків.

\subsection{Природний радіаційний фон}

Однак радіація впливає на нас не лише під час таких масштабних трагедій та їх наслідків. Наприклад в приміщеннях на ґрунтах з високим вмістом радіонуклідів може накопичуватися радіоактивний газ радон. Невеликі дози опромінення протягом тривалого часу так само шкодять клітинам організму. Як правило, цей вплив настільки малий, що клітини встигають відновлюватися. А якщо їм це не вдається, то наслідки опромінення все одно можуть не проявлятися десятиліттями. Та чим вища доза опромінення, тим вищий ризик появи хвороб – деяких видів раку (наприклад, лейкемії), генетичних мутацій, проблем з репродуктивною системою. Особливо чутливі до радіації діти та підлітки, вагітні та жінки загалом.

На щастя, природний радіаційний фон зовсім незначний, тому імовірність зазнати від нього шкоди дуже маленька. Вплив радіації на організм прийнято вимірювати у Зівертах – це величина, що враховує не лише кількість випромінювання, а також чутливість тканин та органів живого організму. За даними ВООЗ, гостра променева хвороба з’являється від опромінення приблизно в 1 Зв, а ризик захворіти на рак значно зростає після 50-100 мЗв (1 мЗв = 0,001 Зв). В той час як в середньому з усіх зовнішніх джерел людина отримує 6,2 мЗв радіації за рік. Та все ж варто бути обачними та не зловживати походами до рентгенологічного кабінету. Рентгенівські промені – це також радіоактивне випромінення, хоча одна флюорографія спричиняє шкоди всього на 20 мкЗв (0,00002 Зв).

\subsection{Користь?}
Як не дивно, та вбивчий вплив радіації на живі тканини може приносити і користь. Опромінення невеликими дозами в медичних цілях – так звана променева або ж радіотерапія – використовують у лікуванні онкологічних захворювань. Її використовують як окремо так і в поєднанні з хіміотерапією чи хірургічним видаленням пухлин. Таке лікування може мати побічні ефекти, пов’язані із впливом радіації (нудоту, головний біль, слабкість), однак, як правило, вони минають із часом.

Використання променевої терапії також дало змогу краще дослідити вплив радіації на здоров’я. Наприклад, за даними американського Національного інституту раку, опромінення може погіршити мисленнєві здібності. Після лікування пухлин мозку – для них радіотерапія часто ефективніша за хіміотерапію чи операції – у дітей спостерігалося зниження коефіцієнту інтелекту (IQ). Їм було складніше здобувати знання та навички, обробляти інформацію, виникали труднощі з пам’яттю та увагою. Це не означає, що ми маємо відмовитися від ефективного лікування, яке рятує життя. Такі дані лише демонструють іще один наслідок тривалого впливу радіації.

\subsection{Зони особливого ризику}

В зоні особливого ризику знаходяться люди, що працюють у пов’язаних із радіацією сферах: ті ж лікарі, що займаються променевою терапією, екіпажі літаків (вони більше піддаються впливу космічного випромінювання), працівники ядерної галузі. Звісно, для них існують спеціальні заходи безпеки, рівень опромінення постійно вимірюється та після перевищення певної дози їм доводиться робити перерву чи взагалі припиняти роботу. Працівники лікарень, де займаються променевою терапією, за даними Міжнародної агенції з атомної енергетики, дотримуючись усіх пересторог, отримують не більше 1 мЗв/рік. Та інші дані не такі оптимістичні. Різноманітні дослідження в атомній сфері показали, що робота в цій галузі загрожує підвищеною імовірністю померти від раку. Особливо вразливі люди, що займаються видобутком уранової руди – радіоактивної речовини, що слугує сировиною для ядерного палива. Вони піддаються не лише зовнішньому опроміненню, а й внутрішньому, вдихаючи радіоактивний газ радон та урановий пил у шахтах.

Внутрішнє опромінення, до речі, відбувається не лише через дихання і не лише із працівниками атомної галузі. Радіонукліди можуть потрапляти в організм із забрудненою їжею чи водою. Наприклад, внаслідок аварії на Чорнобильській АЕС, у довкілля потрапив радіоактивний Цезій-137, який знаходитиметься тут ще протягом сотні років. Видобування, підняття на поверхню, збагачення і обробка уранової руди так само продукує викиди нуклідів, що можуть переноситись повітрям. Потрапляючи в організм із їжею, водою чи повітрям, вони відкладаються у кістках і м’язах і опромінюють людину зсередини.

Крім того, сама атомна станція навіть в процесі нормальної роботи викидає в довкілля радіоактивні речовини, зокрема, невелику кількість радіоактивних газів з приміщень. Також АЕС продукують радіоактивні відходи, для яких досі не існує технології, яка дозволила б зробити їх безпечними зараз чи зберігати протягом усього періоду, поки вони становитимуть загрозу.

\subsection{Аварії та ії настідки}

Проблема із впливом радіації на здоров’я ще й у тому, що він може проявитися одразу – як у випадку з аваріями чи вибухами – або протягом невизначеного періоду часу «потім». «Ризик захворювання на рак» означає, по суті, що його можуть діагностувати завтра, а можуть – ніколи. Якщо додати до цього те, що радіацію не помітити жодним із чуттів, і те, що для багатьох це явище незнайоме, невідоме і незрозуміле, то стає помітним іще один вплив радіації на здоров’я – психологічний. За даними ВООЗ, після ядерних інцидентів чи надзвичайних ситуацій можливий розвиток коротко- або довгострокових психологічних розладів через потенційний вплив радіації. Ментальне здоров’я людини може постраждати навіть якщо радіація її ніяк не торкнеться, адже навіть аварія на атомній станції на іншому боці планети може змусити хвилюватися за своє майбутнє людину, що мешкає неподалік АЕС.

Перша рекомендація ВООЗ для подолання тривожності під час ядерних аварій чи інцидентів – надавати людям, на яких вони можуть вплинути, зрозумілу інформацію про ризики для здоров’я та чіткі інструкції, що необхідно робити. Пам’ятаючи, що після найбільшої в історії катастрофи на АЕС інформація про неї не висвітлювалася в Україні ще протягом двох днів, може бути складно довіряти державі. Тож відсутність повідомлень про інциденти на атомних станціях не завжди позбавляє людей тривожності.

\subsection{Ми незахищені і це норма}

Радіоактивне випромінювання може впливати на організм раптовими великими дозами чи потроху протягом тривалого часу. Воно може нашкодити клітинам організму, і вони розвинуться у рак або навпаки відновляться, так що людина навіть нічого не помітить.
Навіть із дотриманням усіх норм та правил, за звичайної роботи атомна енергетика викидає радіонукліди під час видобутку палива, продукує радіоактивні відходи, що будуть небезпечними ще протягом тисячоліть. А ще вона не дає жодних гарантій, що і далі працюватиме «звичайно», без жодних інцидентів. Особливо враховуючи те, що в Україні вже 12 із 15 енергоблоків вичерпали свій проєктний термін і працюють понаднормово, а відновлювальна енергетика стала дешевшим джерелом енергії. Настав час визнати, що від атомної енергетики більше проблем, ніж рішень, і почати готуватися до безпечного закриття старих атомних енергоблоків, переходити на відновлювану енергетику та впроваджувати енергоефективні заходи.

\section{Дозометрія}
\subsection{Що це?}
Дозиметрі?я — самостійний розділ прикладної ядерної фізики, який розглядає фізичні величини, що характеризують поле іонізуючого випромінювання та взаємодію випромінювання з речовиною, а також принципи і методи визначення цих величин. Вона має справу з такими фізичними величинами, які пов'язані з очікуваним радіаційним ефектом. Важливою задачею дозиметрії є визначення дози випромінювання в різних речовинах, насамперед у тканинах живого організму для виявлення, оцінки і попередження будь-якої можливої радіаційної небезпеки для людини, та для розробки спеціальних засобів і методів радіаційного захисту.

\subsection{Види дозометрій}

\begin{enumerate}
	\item Клінічна дозиметрія
	\begin{itemize}
        \item Клінічна дозиметрія насамперед є невід'ємною частиною променевої терапії, а також має місце при діагностичних процедурах, де використовується іонізуюче випромінювання (наприклад, комп'ютерна томографія, позитрон-емісійна томографія). Основною метою клінічної дозиметрії є вибір оптимального просторово-часового розподілу поглиненої енергії випромінювання в тілі пацієнта, і кількісний опис цього розподілу.
    \end{itemize}
    \item Дозиметрія навколишнього середовища
	\begin{itemize}
        \item Природний радіаційний фон,  радіаційному навантаженню від якого зазнає все населення планети, виникає через космічне випромінювання та природні радіоактивні нукліди, що знаходяться у верхніх шарах земної кори. \newline Базові значення:
        \begin{itemize}
            \item на вулиці (відкритій місцевості) — 8-12 мкР/год;
            \item в приміщенні — 15-30 мкР/год; 
            \newline Допустима норма радіаційного фону у приміщеннях — до 50 мкР/год. \newline Норми об'ємної активності радону:
            \item для експлуатації новобудов не більше 50 Бк/м³;
            \item для старих будинків не більше 100 Бк/м³; 
            \newline Гранична допустима концентрація 400 Бк/м³.
        \end{itemize}
    \end{itemize}
    \item Технічна (промислова) дозиметрія
    \begin{itemize}
    \item В галузях промисловості, де використовується іонізуюче випромінювання різних типів або реакції ядерного розпаду, для оцінки параметрів отриманих полів, поглинутої дози зразків, або продуктів ядерного розпаду використовується технічна дозиметрія, яка оперує методами визначення дози для сильних радіаційних полів (напиклад, при радіаційній стерилізації зразок накопичує дозу 30—60 кГр)
    \newline До галузей промисловості, що використовують методи дозиметрії можна віднести:
    \begin{itemize}
        \item ядерну енергетику;
        \item стерилізацію продуктів харчування, медичних інструментів, фармацевтичних препаратів, пастеризацію;
        \item дезінфекцію (знищення бактерій та продуктів біологічного забруднення), дезінсекцію;
        \item очистку газів та стічних вод від шкідливих домішок;
        \item радіаційне зшивання полімерів, радіаційна полімеризація;
        \item наукові дослідження;
        \item інші сфери виробництва.
    \end{itemize}
    \end{itemize}
\end{enumerate}

\subsection{Небезпечні дози однократного загального опромінення}

Опромінення — дія різних видів випромінювання на організм людини і тварин, що викликають різноманітні біохімічні та фізико-хімічні реакції. Всі вони поділяються на корпускулярні та фотонні, а останні, у свою чергу, — на низько- та високоенергетичні (син.: тонізувальні). Найбільш значними вони є під впливом іонізуючого типу, до якого відносять будь-яке О., взаємодія якого з речовиною призводить до виникнення електричних зарядів протилежних знаків. О. використовується в медицині для променевої терапії та діагностики захворювань, а також може виникати при аварійних ситуаціях, при виготовленні та використанні джерел іонізуючого випромінення.

\begin{table}[htb]
\centering
\begin{tabular}{l|r}
Наслідки & Опромінення \\\hline
Загибель окремих клітин крові та полових клітин & 0,1-0,5 Зв (10-50 бер) \\
Порушення у роботі кровотворної системи & 0,5-1 Зв (50-100 бер)\\
Гостра променева хвороба (50 відсотків смертних випадків) & 3-5 Зв (300-500 бер)\\
\end{tabular}
\caption{\label{tab:widgets}Таблиця небезпечних доз.}
\end{table}

\begin{thebibliography}{3}
\bibitem{book1} https://uk.wikipedia.org/wiki/Радіація
\bibitem{book2} https://uk.wikipedia.org/wiki/Опромінення
\bibitem{book3} https://uk.wikipedia.org/wiki/Дозиметріяid=77154&chapterid=37031
\bibitem{book4} http://cde.nuft.edu.ua/mod/book/view.php?id=77154&chapterid=37031
\bibitem{book5} https://www.latex-project.org/help/documentation/
\end{thebibliography}

\end{document}